\documentclass[xcolor=svgnames,smaller]{beamer}
\usepackage[utf8]{inputenc}
\usepackage[T1]{fontenc}
\usepackage[english]{babel}
\usepackage{xspace}
\usepackage{amsmath}
\usepackage{amsthm}
\usepackage{mathtools}
\usepackage{bm}
\usepackage{booktabs}
\usepackage{microtype}
\usepackage[version=3]{mhchem}
\usepackage{siunitx}

\usepackage{tikz}
\usetikzlibrary{decorations.pathmorphing}
\tikzset{font=\small} % Samme fontstørrelse i TikZ som i figurtekst
\usepackage{pgfplots}
\pgfplotsset{compat=1.6}

\usepackage[sc]{mathpazo}
\renewcommand{\ttdefault}{txtt}
\usepackage{url}
\urlstyle{rm}

% itemize alignment: http://tex.stackexchange.com/questions/14038/tabbing-inside-itemize-or-itemize-inside-tabbing
\usepackage{listliketab}


%% Custom commands:

% Afledte, partielle og ej, der alle tager to argumenter (som
% \frac). \diff er almindeligt afledt én gang og \ddiff er to gange -
% begge med oprejste(!) hårde d'er. \pdiff er partielt afledt én gang
% og \ppdiff to gange.
\newcommand\pdiff[3][\partial]{\frac{#1 #2}{#1 #3}}
\newcommand\ppdiff[3][\partial]{\frac{#1^2 #2}{#1 #3^2}}
\newcommand\ddiff[2]{\pdiff[\mathrm{d}]{#1}{#2}}
\newcommand\dddiff[2]{\ppdiff[\mathrm{d}]{#1}{#2}}

% Kommando til at lave parenteser på begge sider
\newcommand\paren[1]{\left(#1\right)}

% Random convenience.
\newcommand{\inv}{^{-1}}             % "I minus første"
\newcommand{\idx}[1]{_{\textup{#1}}} % Tekst-indeks med få tastetryk.
\newcommand{\order}[1]{^{\textup{(#1)}}} % Orden
\newcommand{\half}{\frac{1}{2}}


\mode<presentation>{
  \usetheme[compress]{Singapore}
  \usecolortheme[named=MidnightBlue]{structure}
  \useinnertheme{circles}
  \usecolortheme{orchid}
  \setbeamertemplate{blocks}[rounded]
  \usefonttheme{serif}
  \usefonttheme{professionalfonts}
  \usefonttheme[onlylarge]{structurebold}
  \setbeamerfont{block title}{series*=m}
  \setbeamercolor{example text}{fg=Maroon}
}

\title{Nonlinear optics}
\subtitle{Wave mixing and the Kerr effect}
\author{
  Anders Aspegren Søndergaard \\
  Kristoffer Theis Skalmstang \\
  Michael Munch \\
  Steffen Videbæk Fredsgaard \\
}
\date{March 12, 2014}

%------------------------------------------------------------------------
% Set depth of table of content
%------------------------------------------------------------------------
\setcounter{tocdepth}{1}

% \AtBeginSection[] {
%   \begin{frame}<beamer>
%     \frametitle{Outline}
%     \tableofcontents[currentsection]
%   \end{frame}
% }

\begin{document}

% \begin{frame}
%   \frametitle{Typografitest}
%   Tekst
%   \begin{itemize}
%   \item Punkt
%   \end{itemize}
%   \begin{block}{block}
%     Hest
%   \end{block}
%   \begin{example}
%     example
%   \end{example}
% \end{frame}
% \end{document}

\frame{\titlepage}
\section{Nonlinear optics}





\begin{align*}
  \label{eq:wave-general}
  \nabla^2 \mathbf E - \frac{1}{c^2} \ppdiff{\mathbf E}{t}
  = \frac{1}{\epsilon_0c^2} \ppdiff{\mathbf P}{t},
\end{align*}

\[
\mathbf P = \epsilon_0 ( \chi\order1 \mathbf E +
\chi\order2 \mathbf E^2 + \chi\order3 \mathbf E^3 \dots )
\]







%%% Local Variables: 
%%% mode: latex
%%% TeX-master: "nonlinearslides"
%%% End: 

\chapter{Three-wave mixing}
\label{cha:mixing}


The Three-wave mixing is a property of the second order nonlinearity of the polarity dependence of the electric field, $\mathcal{E}(t)$. 
It is a phenomenon where the two waves mix together to a third. 



\section{Mathematical formulation and uses}
\label{sec:mixing-math}

To qualitatively examine the interaction between the two electromagnetic waves, of frequencies $\omega_1$ and $\omega_2$ with electric field amplitude $\mathcal{E}_1  \equiv \mathcal{E}(\omega_1)$ and $\mathcal{E}_1  \equiv \mathcal{E}(\omega_1)$ respectively, a superposition of these monochromatic plane waves is composed: 
\[
\mathbf{\mathcal{E}}(t) = \Re (\mathcal{E}_1e^{i(\mathbf{k_1} \cdot \mathbf{r} - \omega_1 t)}+\mathcal{E}_2e^{i(\mathbf{k_2} \cdot \mathbf{r} \omega_2 t)}),
\]
and put into the second order part of the polarity expansion \cite[sec.~21.2C]{saleh}
\begin{align}
     \mathbf{P}^{NL} & = \varepsilon_0 \chi^{(2)} \mathbf{E}^2(t) \nonumber \\
&= \varepsilon_0 \chi^{(2)} 2 \Re \left[
\left(|\mathcal{E}_1|^2+|\mathcal{E}_2|^2\right)e^{0} + |\mathcal{E}_1|^2e^{i2(\mathbf{k_1} z - \omega_1t)}+|\mathcal{E}_2|^2e^{i2(\mathbf{k_2} z - \omega_2t)} \right.\nonumber \\
& \left.
+ 2\mathcal{E}_1 \mathcal{E}_2e^{i((\mathbf{k_1} + \mathbf{k_2}) \cdot \mathbf{r} - (\omega_1+\omega_2)t)} 
+ 2\mathcal{E}_1 \mathcal{E}_2^*e^{i((\mathbf{k_1} - \mathbf{k_2}) \cdot \mathbf{r} - (\omega_1-\omega_2)t)} 
\right] \label{eq:MixPNL}.
\end{align}
Since this gives feedback to the electromagnetic field in \cref{eq:wave-general}, new frequencies are generated; $0$, the uninteresting one, $2\omega_1$ and $2\omega_2$, which are the original frequencies second harmonics, and more interesting, the sum and difference of the initial frequencies. The last two frequency couplings are often referred to as up- and down-conversion respectively.

It is clear that for the up- and down-conversion the third wave frequency will be:
\begin{equation}
  \label{qe:freqCond}
\omega_3 = \omega_1 \pm \omega_2
\end{equation}
and that a further phase matching constraint on the wave vector is necessary,
\[
\mathbf{k}_3 = \mathbf{k}_1 \pm \mathbf{k}_2,
\]
for a strong feedback in electric field change.

Physical this means that ones the third wave is created, it will also couple back to
the two other waves, but these constraints secures that a neutral interaction sustained through time and space. 




\section{Numeric formulation of up-conversion}
\label{sec:mixing-numeric}

To get a numerical interpretation of the result, one can assume the
\cref{qe:freqCond} and inject the up-conversion part of \cref{eq:MixPNL} into the time dependent wave equation of each wave \cite[Equation 3.39]{shen}:

\begin{align*}
\paren{ \pdiff{}{z}  + \frac{1}{v_i^{g}} } \mathcal{E}_i \paren{z, t}
&= \frac{i 2 \pi \omega_i^{2}}{k c^2} \mathbf{P}^{NL} \paren{z,t} e^{i \paren{k_i z - \omega_{i}
    t}} \Rightarrow \tag{3.39}\\
\paren{ \pdiff{}{z}  + \frac{1}{v_i^{g}} } \mathcal{E}_i \paren{z, t}
&= \frac{i \omega_i \chi^{(2)}}{2 n_i c} \mathcal{E}_j^{*}\mathcal{E}_k e^{i (k_3 - k_2 -k_1) z
  t}, \text{where } i \neq j \neq k \\
% \paren{ \pdiff{}{z}  + \frac{1}{v_1^{g}} } \mathcal{E}_1 \paren{z, t}
% &= \frac{i \omega_1 \chi^{(2)}}{2 n_1 c} \mathcal{E}_2^{*}\mathcal{E}_3 e^{i (k_3 - k_2 -k_1) z
%     t} \\
% \paren{ \pdiff{}{z}  + \frac{1}{v_2^{g}} } \mathcal{E}_2 \paren{z, t}
% &= \frac{i \omega_2 \chi^{(2)}}{2 n_2 c} \mathcal{E}_1^{*}\mathcal{E}_3 e^{i (k_3 - k_2 -k_1) z
%     t} \\
%   \paren{ \pdiff{}{z}  + \frac{1}{v_3^{g}} } \mathcal{E}_3 \paren{z, t}
%   &= \frac{i \omega_3 \chi^{(2)}}{2 n_3 c} \mathcal{E}_1\mathcal{E}_2 e^{i (k_3 - k_2 -k_1) z
%     t}, \\
\end{align*}
where the identity $\omega_i = k_ic$ has been used, and also the rotating wave
approximation and slowly varying amplitude approximation. These are equivalent to
the coupled partial differential
equations  (2) in \cite{bakker} and can be solved with the Runge-Kutta procedure
as described.

Typically this is done in KDP or KTP crystals such as the sum frequency conversion
done in \cite{sumFreq} where a 455 nm is med from a 807 nm and a 1062 source. Though only with an efficienty of $0.01\%$.

Multiple wave mixing can be optained at higher orders of nonlinearity terms, such
that the third order gives rise to four-wave mixing, but these will naturaly become
even more restrained. 

%%% Local Variables: 
%%% mode: latex
%%% TeX-master: "nonlinear"
%%% End: 



\section{The Kerr effect}
\begin{frame}
  \frametitle{Kerr effect: Mathematical formulation}

  With a centrosymmetric and isotropic medium the important term is
  \[
  \mathbf P = \epsilon_0 ( \chi\order1 \mathbf E +
  {\chi\order2 \mathbf E^2} + \textcolor{red}{\chi\order3 \mathbf E^3} \dots )
  \]
\end{frame}

\begin{frame}
  \frametitle{Suspectibility}

  The refractive index of the medium will then be intensity dependent. 
  \begin{align*}
    n &= \sqrt{1 + \chi_1 + \chi_3 |\mathcal{E}|^2}\\
    &\simeq n_0 + \tfrac{1}{2} n_2 |\mathcal{E}|^2,
  \end{align*}
\end{frame}

\begin{frame}
  \frametitle{Gaussion beams}
  Assume: Beam has a gaussian shape

  $\Rightarrow$ Refractive index is peaked at center.
  {\centering
  \includegraphics[width=1\columnwidth]{gaussrefrac}}

This is effectively a positive lens that will focus the beam. 
\end{frame}

\begin{frame}
  \frametitle{Waveequation}
  Assume: Monocromatic linear polarized plane wave.
  \[E = \mathcal{E}(x,y) e^{-i (\omega t-kz)}\]
  Insert into to the wave equation
\[  \nabla^2 E - \frac{1}{c^2} \ppdiff{E}{t}
  = \frac{1}{\epsilon_0c^2} \ppdiff{P}{t}\]
  This gives the nonlinear wave equation:

  \[\nabla^2 \mathcal{E} + \frac{\omega^{2}}{c^2} \mathcal{E} = - \frac{\omega^2}{\epsilon_0 c^2} \left( \epsilon_0 \chi_1
    \mathcal{E} + \epsilon_0 \chi_3 |\mathcal{E}|^2 \mathcal{E}
  \right).\]

  which is a partial differential equation !
\end{frame}


%%% Local Variables: 
%%% mode: latex
%%% TeX-master: "nonlinearslides"
%%% End: 

\begin{frame}
  \frametitle{Usage of the Kerr effect}
  
\end{frame}


%%% Local Variables: 
%%% mode: latex
%%% TeX-master: "nonlinearslides"
%%% End: 

\begin{frame}
  \frametitle{Solving partial differential equations}
  It's a tough job!

  \includegraphics[width=\columnwidth]{putin}
\end{frame}

\begin{frame}{Options}
\begin{enumerate}
\item Crank-Nicholson
\item Iterative mesh techniques

  Update grid untill PDE is fullfilled in all points. 
\end{enumerate}
\end{frame}


\begin{frame}
  \frametitle{Crank-Nicholson}
  Applies for diffusion problems:
  \[  \pdiff{u}{t} = F \Big(u, x, t, \pdiff{u}{x}, \ppdiff{u}{x} \Big). \]
  Utilize the equality of forward- and backwards Euler method.
  \begin{center}
    \includegraphics[width=0.4\columnwidth]{Crank}
  \end{center}
\end{frame}

\begin{frame}{Finite difference}

Use discrete version of differential operators

\begin{align*}
  f'(x) &= \frac{f(x+\half h) - f(x - \half h)}{h}\\[5mm]
  f''(x) &= \frac{f(x+h) - 2f(x) + f(x-h)}{h^{2}}.
\end{align*}

Introduce unitless quantities and assume cylindrical beam
\[   i \pdiff{\tilde E}{\tilde z} + \ppdiff{\tilde E}{\tilde r}
+ \frac{1}{\tilde r} \pdiff{\tilde E}{\tilde r}
+ |\tilde E|^2 \tilde E
= 0,\]
  
\end{frame}

\begin{frame}
  \frametitle{End result}
  A "almost" five diagonal matrix system.
  $n$ is z direction, $i$ is radial direction
  
  \begin{align*}
    (\alpha + 2\beta^{2} - |E_{i}^{n+1}|^{2})E_{i}^{n+1} - \beta^{2}(E_{i+2}^{n+1} + E_{i-2}^{n+1}) -
    \frac{\beta}{r_{i}} (E_{i+1}^{n+1} - E_{i-1}^{n+1}) \\
    = (\alpha - 2\beta^{2} + |E_{i}^{n}|^{2})E_{i}^{n} + \beta^{2}(E_{i+2}^{n} + E_{i-2}^{n}) +
    \frac{\beta}{r_{i}} (E_{i+1}^{n} - E_{i-1}^{n}),
  \end{align*}


  By approximating $|E_{i}^{n+1}|^{2} \approx |E_{i}^{n}|^{2}$ this is a matrix problem.

  This must be solved for every step!


  \vfill
  $\alpha = -2i/\!{\Delta z}$ and $\beta = 1/{2 \Delta r}$.

\end{frame}

\begin{frame}
  \frametitle{Well....}
  We could not fix numerical problems in $r = 0$. \bigskip

  \includegraphics[width=0.45\columnwidth]{kerr_double} \hfill \includegraphics[width=0.45\columnwidth]{kerr_single}

  The iterative mesh technique requires us to solve a $N_{z} \times N_{r}$ set of nonlinear
  equations.

  We didn't have time to do this....
\end{frame}
%%% Local Variables: 
%%% mode: latex
%%% TeX-master: "nonlinearslides"
%%% End: 



\end{document}

%%% Local Variables: 
%%% mode: latex
%%% TeX-master: t
%%% TeX-engine: default
%%% End:
