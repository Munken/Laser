\chapter{The Kerr effect}
\label{cha:kerr}

The Kerr effect is found in media, where the index of refraction is
dependent on the electric field of the propagating light. For real
life laser beams, where the (time avaraged) amplitude of electric
field depends on the position in the beam, this leads to a index of
refraction varying over the spacial position within the beam. Under
certain and experimentally obtainable conditions, this leads to
self-focusing of the beam within a passive medium.


\section{Mathematical formulation}
\label{sec:kerr-math}

As for second harmonic generation, which has been described in the
course, our model is a two-state atom and a monochromatic field with
semi-classical light/matter interaction. As in \cite{milonni}, the
frequency of monochromatic laser field is determined by the energies
of the two states of the atom, $\omega = \frac{E_2 - E_1}{\hbar}$,
leading to the expression
\begin{align}
  \label{eq:kerr-laser-field}
  E = \mathcal{E}(r, \omega) e^{-i \omega t},
\end{align}
for the electric field in a isotropic material with radial symmetry;
here $\mathcal{E}$ is the complex field amplitude. The polarization
can then be described similarily as $\mathcal{P}(r, \omega) e^{-i
  \omega t}$, where strength of the polarization $\mathcal{P}$ has
both a linear and a non-linear contribution.

In this formulation, the wave equation becomes
\begin{align}
  \label{eq:kerr-wave-eqn}
  \nabla^2 \mathcal{E} + \frac{\omega^2}{c^2} \mathcal{E}
  = \frac{1}{\epsilon_0 c^2} \mathcal{P}
\end{align}





%%% Local Variables: 
%%% mode: latex
%%% TeX-master: "nonlinear"
%%% End: 
