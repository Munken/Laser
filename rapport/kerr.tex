\chapter{The Kerr effect}
\label{cha:kerr}

The Kerr effect is found in media, where the index of refraction is
dependent on the electric field of the propagating light. For real
life laser beams, where the (time avaraged) amplitude of electric
field depends on the position in the beam, this leads to a index of
refraction varying over the spacial position within the beam. Under
certain and experimentally obtainable conditions, this leads to
self-focusing of the beam within a passive medium.


% \section{Mathematical formulation}
% \label{sec:kerr-math}

% As for second harmonic generation, which has been described in the
% course, our model is a two-state atom and a monochromatic field with
% semi-classical light/matter interaction. The frequency of
% monochromatic laser field is determined by the energies of the two
% states of the atom, $\omega = \frac{E_2 - E_1}{\hbar}$. In the
% following we follow \textcite[][sec. 10.3]{milonni}, which gives the
% expression
% \begin{align}
%   \label{eq:kerr-laser-field}
%   E = \mathcal{E}(r, \omega) e^{-i \omega t},
% \end{align}
% for the electric field in a centrosymmetric (ie. the quantum states
% have definite parity) and isotropic material; here $\mathcal{E}$ is
% the complex field amplitude. The polarization can then be described
% similarily as $\mathcal{P}(r, \omega) e^{-i \omega t}$, where strength
% of the polarization $\mathcal{P}$ has both a linear and a nonlinear
% contribution.

% In this formulation, the wave equation becomes
% \begin{align}
%   \label{eq:kerr-wave-eqn}
%   \nabla^2 \mathcal{E} + \frac{\omega^2}{c^2} \mathcal{E}
%   = - \frac{\omega^2}{\epsilon_0 c^2} \mathcal{P},
% \end{align}
% where the lowest-order nonlinear polarization is
% \begin{align}
%   %\nonumber
%   \mathcal{P} &= \mathcal{P}^{\textup L} + \mathcal{P}^{\textup{NL}}
%   %\\
%   \label{eq:kerr-polarization}
%   = \epsilon_0 \chi_1 \mathcal{E}
%   + \epsilon_0 \chi_3 |\mathcal{E}|^2 \mathcal{E}.
% \end{align}
% It should be noted, that the succeptibilities here are not simple
% scalars, but are certain values of the nonlinear succeptibility
% tensor, which is discussed in \textcite[sec.~10.2]{milonni}. In
% our model using a monochromatic field, $\chi_1$ and $\chi_3$ are
% constant as $\omega$ is constant.

% When the field intensity is not large enough to require higher than
% third order terms in the polarization, the polarization term of the
% wave equation~\eqref{eq:kerr-wave-eqn} becomes
% \begin{align}
%   \label{eq:kerr-wave-pol}
%   - \chi_1 \frac{\omega^2}{c^2} \mathcal{E}
%   - \chi_3 \frac{\omega^2}{c^2} |\mathcal{E}|^2 \mathcal{E}
%   \equiv - \chi \frac{\omega^2}{c^2} \mathcal{E}.
% \end{align}
% Permittivity, succeptibility, and refractive index i related as $n^2 =
% \epsilon = 1 + \chi$ and we find that
% \begin{align}
%   n &= \sqrt{1 + \chi_1 + \chi_3 |\mathcal{E}|^2}
%   = n_0 \sqrt{1 + \frac{\chi_3}{n_0^2} |\mathcal{E}|^2}
%   \simeq n_0 + \frac{\chi_3}{2n_0} |\mathcal{E}|^2
%   \\
%   \label{eq:kerr-n-chi}
%   &\equiv n_0 + \tfrac{1}{2} |\mathcal{E}|^2 n_2,
% \end{align}
% where $n_0 = \sqrt{1 + \chi_1}$ is the linear refractive index in the
% medium. Defining $E = \tfrac{1}{2}( \mathcal{E} e^{-i \omega t} +
% \text{c.c.})$ we arrive at the Kerr nonlinearity
% $
% %\begin{align}
% %  \label{eq:kerr-nonlin}
% n = n_0 + n_2 E^2,
% %\end{align}
% $ which is an approximation, where we have discarded terms of order
% higher than two (only even orders appear in the full expression). We
% are assured by \textcite{milonni}, that this is indeed an excellent
% approximation.


\section{Mathematical details at a glance}
\label{sec:kerr-math}

The mathematical formulation of the Kerr effect is derived in
\textcite[sec.~10.3]{milonni}. For a monochromatic laser field $E =
\mathcal{E} e^{-i \omega t}$ with frequency $\omega$ in an isotropic
and centrosymmetric medium (ie. a medium which quantum states have
definite parity), an alternative wave equation for the complex laser
field and polarization amplitudes is found. The most important point
here is, that the polarization has a linear and a nonlinear
contribution $\mathcal{P} = \mathcal{P}^{\textup L} +
\mathcal{P}^{\textup{NL}}$:
\begin{align}
  \label{eq:kerr-wave-eqn}
  \nabla^2 \mathcal{E} + \frac{\omega^2}{c^2} \mathcal{E}
  &= - \frac{\omega^2}{\epsilon_0 c^2} \mathcal{P}
  %\\
  = - \frac{\omega^2}{\epsilon_0 c^2} \left( \epsilon_0 \chi_1
    \mathcal{E} + \epsilon_0 \chi_3 |\mathcal{E}|^2 \mathcal{E}
  \right).
\end{align}
It should be noted, that the succeptibilities here are not simple
scalars, but are certain values of the nonlinear succeptibility
tensor. For a monochromatic field, $\chi_1$ and $\chi_3$ are constant
as $\omega$ is constant. Defining \fxwarning{Implicit approksimation:
  feltet er ikke så stærkt, at der kræves højere ordener.}
\begin{align}
  - \chi \frac{\omega^2}{c^2} \mathcal{E}
  \equiv - \chi_1 \frac{\omega^2}{c^2} \mathcal{E}
  - \chi_3 \frac{\omega^2}{c^2} |\mathcal{E}|^2 \mathcal{E}
\end{align}
and using $n^2 = 1 + \chi$ leads to
\begin{align}
  n &= \sqrt{1 + \chi_1 + \chi_3 |\mathcal{E}|^2}
  = n_0 \sqrt{1 + \frac{\chi_3}{n_0^2} |\mathcal{E}|^2}
  \\
  &\simeq n_0 + \frac{\chi_3}{2n_0} |\mathcal{E}|^2
  \label{eq:kerr-n-chi}
  \equiv n_0 + \tfrac{1}{2} n_2 |\mathcal{E}|^2,
\end{align}
where $n_0 = \sqrt{1 + \chi_1}$ is the linear refractive index in the
medium. Clearly, the effective refractive index felt by the
propagating wave, depends on the form of the electric field.


\section{Qualitative interpretation and uses}
\label{sec:kerr-interpr}

As shown i figure~\ref{fig:kerr-refrac}, a gaussian beam, which is a
beam exhibiting cylindrical symmetry and where the intensity
distribution is or nearly is gaussian, feels a larger refractive index
closer to the beam axis. Eqn.~\eqref{eq:kerr-n-chi} shows, that this
nonlinear correction is proportional to the norm square of the field
amplitude, ie. the beam intensity.

The beam size of an gaussian beam is not constant, but varies along
the beam axis $z$, with minimum beam cross section found at the beam
waist. Using a geometric interpretation of the light, rays moving away
from the waist have increasing distance from the axis, ie. radial
distance $r$, leading to a broadening of the beam and a lower
intensity. As such, the rays move from a region of high to lower
refraction index, leading to a reflection of the ray back towards the
beam axis.

\begin{figure}[t!]
  \centering
  \fbox{\begin{minipage}[t][3cm]{.3\linewidth}
      \centering\vspace*{0pt plus 1 fill}
      Woop woop!
      \vspace*{0pt plus 1 fill}
    \end{minipage}}
  \caption{Gaussisk beamprofil og effektivt brydningsindeks.}
  \label{fig:kerr-refrac}
\end{figure}

This self-focusing competes with diffractive spreading and below a
certain critical intensity, self-focusing is not observed. This limit
is normally given in terms of total beam power, for example:
\begin{align}
  \label{eq:1}
  P\idx{crit}^{\textup{gauss}}
  &\sim \frac{c \epsilon_0}{8 \pi} \frac{\lambda^2}{n_2}
  &&\text{Gaussian beam \cite{milonni}},
  \\
  P\idx{crit}^{\textup{cyl}}
  &\approx \frac{c a^2}{64} \frac{\lambda^2}{n_2}
  &&\text{Cylindrical beam of uniform intensity \cite{prl-selftrap}},
\end{align}
where the angular diffractive divergence of the cylindrical beam is $a
\lambda / n_0 2 r$. Note here, that the critical power decreases
linearly in $n_2$ but increases quadratically in wavelength.

As can be expected, extreme power past the critical power will lead to
undesired effects in the medium. These effects include multiphoton
ionisation and other dielectric breakdown effects, potentially leading
to destruction of the medium.

In optics, media where this self-focusing is present are termed Kerr
lenses, as the passive medium acts as a lens for beams passing
through. The dioptric power (inverse focal length) of a Kerr lens of
length $z$ for a circular gaussian beam of power $P$ with waist radius
$w_0$ is
\cite{yefet-kerrlens}
\begin{align}
  \label{eq:kerr-focal-length}
  f\idx{kerr}\inv = \frac{4}{\pi} \frac{n_2 P}{w_0^4} z.
\end{align}

Kerr lenses have the added anvantage over standard optical lenses,
that the Kerr lenses do not require alignment of the optical axis of
the lens and the beam axis. The is due to the fact, that the optical
axis of the Kerr lens is induced by the electric field of the laser
beam. This advantage is utilized in Kerr lens mode-locking.

\section{Numerical work}

The partial differential equation given in Eqn.~\eqref{eq:kerr-diff} is categorized as a 1D
diffusion problem. A method to solve this problem is the Crank-Nicolson method which applies for
equations of the form
\begin{equation}
  \label{eq:Crank-Nicolson}
  \pdiff{u}{t} = F \Big(u, x, t, \pdiff{u}{x}, \ppdiff{u}{x} \Big), 
\end{equation}
which is also the case for our equation, since it can be rewritten as
\begin{equation}
  \label{eq:kerr-diff-crank}
  i\pdiff{E}{z} = -\half \left[\frac{1}{r} \pdiff{E}{r} + \ppdiff{E}{r} + |E|^{2} E \right]. 
\end{equation}

In order to evaluate this numerically we discretize $z$ and $r$ in order to make a grid. The first
and second order derivatives can then be approximated by the central finite difference
\begin{equation}
  \label{eq:central-diff}
  f'(x) = \frac{f(x+\half h) - f(x - \half x)}{h} \quad f''(x) = \frac{f(x+h) - 2f(x) + f(x-h)}{h^{2}}.
\end{equation}

The Crank-Nicolson method then relates the value of $E$ at the $n+1$ $z$-step to the value in the
$n$'th step via the equalities of the forward and backward euler method. Note that $i$ denote the
step in $r$
\begin{equation}
  \label{eq:Crank-Nicolson-discrete}
  \frac{E^{n+1}_{i} - E^{n}_{i}}{\Delta z} = \half \big[F_{i}^{n+1}+F_{i}^{n} \big].
\end{equation}
The right hand side of the can be evaluated using the central difference approximation. This gives
a set of equations for $E^{n+1}$ that depends on $E^{n}$ ie. the values in the previous step
\begin{align}
  \notag
  (\alpha + 2\beta^{2} - |E_{i}^{n+1}|^{2})E_{i}^{n+1} - \beta^{2}(E_{i+2}^{n+1} + E_{i-2}^{n+1}) -
  \frac{\beta}{r_{i}} (E_{i+1}^{n+1} - E_{i-1}^{n+1}) \\
  = (\alpha - 2\beta^{2} + |E_{i}^{n}|^{2})E_{i}^{n} + \beta^{2}(E_{i+2}^{n} + E_{i-2}^{n}) +
  \frac{\beta}{r_{i}} (E_{i+1}^{n} - E_{i-1}^{n}), 
  \label{eq:kerr-long}
\end{align}
where $\alpha = -2i/\!{\Delta z}$ and $\beta = 1/{2 \Delta r}$.

This is almost a fivediagonal matrix equation except for the term that is cubic in
$E_{i}^{n+1}$. However with sufficiently small steps one can approximate the cubic term to be
$|E_{i}^{n+1}|^{2}E_{i}^{n+1} \approx |E_{i}^{n}|^{2}E_{i}^{n+1}$ ie. take the quadratic part from
the last step. In each timestep one must then solve the matrix inversion problem
\begin{equation}
  \label{eq:matrix}
  \mathbf{A} \mathbf{E}^{n+1} = \mathbf{B} \mathbf{E}^{n}, 
\end{equation}
where the coefficients that make up these two matricies is the coefficients listed in
Eqn.~\eqref{eq:kerr-long}. 




%%% Local Variables: 
%%% mode: latex
%%% TeX-master: "nonlinear"
%%% End: 
