\chapter{The Kerr effect}
\label{cha:kerr}

The Kerr effect is found in media, where the index of refraction is
dependent on the electric field of the propagating light. For real
life laser beams, where the (time avaraged) amplitude of electric
field depends on the position in the beam, this leads to a index of
refraction varying over the spacial position within the beam. Under
certain and experimentally obtainable conditions, this leads to
self-focusing of the beam within a passive medium.


\section{Mathematical formulation}
\label{sec:kerr-math}

As for second harmonic generation, which has been described in the
course, our model is a two-state atom and a monochromatic field with
semi-classical light/matter interaction. The frequency of
monochromatic laser field is determined by the energies of the two
states of the atom, $\omega = \frac{E_2 - E_1}{\hbar}$. In the
following we follow \textcite[][sec. 10.3]{milonni}, which gives the
expression
\begin{align}
  \label{eq:kerr-laser-field}
  E = \mathcal{E}(r, \omega) e^{-i \omega t},
\end{align}
for the electric field in a centrosymmetric (ie. the quantum states
have definite parity) and isotropic material; here $\mathcal{E}$ is
the complex field amplitude. The polarization can then be described
similarily as $\mathcal{P}(r, \omega) e^{-i \omega t}$, where strength
of the polarization $\mathcal{P}$ has both a linear and a nonlinear
contribution.

In this formulation, the wave equation becomes
\begin{align}
  \label{eq:kerr-wave-eqn}
  \nabla^2 \mathcal{E} + \frac{\omega^2}{c^2} \mathcal{E}
  = - \frac{\omega^2}{\epsilon_0 c^2} \mathcal{P},
\end{align}
where the lowest-order nonlinear polarization is
\begin{align}
  %\nonumber
  \mathcal{P} &= \mathcal{P}^{\textup L} + \mathcal{P}^{\textup{NL}}
  %\\
  \label{eq:kerr-polarization}
  = \epsilon_0 \chi_1 \mathcal{E}
  + \epsilon_0 \chi_3 |\mathcal{E}|^2 \mathcal{E}.
\end{align}
It should be noted, that the succeptibilities here are not simple
scalars, but are certain values of the nonlinear succeptibility
tensor, which is discussed in \textcite[sec.~10.2]{milonni}. In
our model using a monochromatic field, $\chi_1$ and $\chi_3$ are
constant as $\omega$ is constant.

When the field intensity is not large enough to require higher than
third order terms in the polarization, the polarization term of the
wave equation~\eqref{eq:kerr-wave-eqn} becomes
\begin{align}
  \label{eq:kerr-wave-pol}
  - \chi_1 \frac{\omega^2}{c^2} \mathcal{E}
  - \chi_3 \frac{\omega^2}{c^2} |\mathcal{E}|^2 \mathcal{E}
  \equiv - \chi \frac{\omega^2}{c^2} \mathcal{E}.
\end{align}
Permittivity, succeptibility, and refractive index i related as $n^2 =
\sqrt{\epsilon}^2 = 1 + \chi$ and we find that
\begin{align}
  \label{eq:kerr-n-chi}
  n &= \sqrt{1 + \chi_1 + \chi_3 |\mathcal{E}|^2}
  = n_0 \sqrt{1 + \frac{\chi_3}{n_0^2} |\mathcal{E}|^2}
  \simeq n_0 + \frac{\chi_3}{2n_0} |\mathcal{E}|^2
  \\
  &\equiv n_0 + \tfrac{1}{2} |\mathcal{E}|^2 n_2,
\end{align}
where $n_0 = 1 + \chi$ is the linear refractive index in the
medium. Defining $E = \tfrac{1}{2}( \mathcal{E} e^{-i \omega t} +
\text{c.c.})$ we arrive at the Kerr nonlinearity
$
%\begin{align}
%  \label{eq:kerr-nonlin}
n = n_0 + n_2 E^2,
%\end{align}
$ which is an approximation, where we have discarded terms of order
higher than two (only even orders appear in the full expression). We
are assured by \textcite{milonni}, that this is indeed an excellent
approximation.




%%% Local Variables: 
%%% mode: latex
%%% TeX-master: "nonlinear"
%%% End: 
