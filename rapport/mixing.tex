\chapter{Three-wave mixing}
\label{cha:mixing}


Three-wave mixing is an effect arising from the second order
nonlinearity of the polarity $\mathbf P = \epsilon_0 ( \chi\order1 \mathbf E +
\chi\order2 \mathbf E^2 + \dots )$.
In a medium with this type of nonlinearity,
two incoming waves can couple together and form one or more additional waves.
As the two waves induce an oscillating polarization in the material,
the nonlinearity couples the frequencies together
to the sum and difference frequencies.
By the otherwise linear wave equation \eqref{eq:wave-general},
these new polarization modes will couple back to the electric field,
resulting in sum and difference frequency electromagnetic waves.

\section{Mathematical derivation}
\label{sec:mixing-math}

To examine the consequences of the second order nonlinearity,
let two plane waves of frequencies $\omega_1$ and $\omega_2$ with electric field amplitude $\mathcal{E}_1$ and $\mathcal{E}_2$, respectively,
enter the nonlinear medium in superposition: 
\[
     \mathbf{\mathbf{E}}(t) = \Re\left(\mathcal{E}_1e^{i(\mathbf{k_1} \cdot \mathbf{r} - \omega_1 t)}+\mathcal{E}_2e^{i(\mathbf{k_2} \cdot \mathbf{r} - \omega_2 t)} \right).
\]
The second order part of the polarity becomes\cite[sec.~21.2C]{saleh}
\begin{align}
     \mathbf{P}^{NL} & = \varepsilon_0 \chi^{(2)} \mathbf{E}^2(t) \nonumber \\
&= \varepsilon_0 \chi^{(2)} 2 \Re \left[
\left(|\mathcal{E}_1|^2+|\mathcal{E}_2|^2\right)e^{0} + |\mathcal{E}_1|^2e^{i2(\mathbf{k_1} z - \omega_1t)}+|\mathcal{E}_2|^2e^{i2(\mathbf{k_2} z - \omega_2t)} \right.\nonumber \\
& \left.
+ 2\mathcal{E}_1 \mathcal{E}_2e^{i((\mathbf{k_1} + \mathbf{k_2}) \cdot \mathbf{r} - (\omega_1+\omega_2)t)} 
+ 2\mathcal{E}_1 \mathcal{E}_2^*e^{i((\mathbf{k_1} - \mathbf{k_2}) \cdot \mathbf{r} - (\omega_1-\omega_2)t)} 
\right] \label{eq:MixPNL}.
\end{align}
It is seen that new frequencies are generated: $0$, the uninteresting\footnote{Being constant, it drops out of the wave equation and produces no radiation.} one, $2\omega_1$ and $2\omega_2$, which are second harmonics of the original frequencies, and more interestingly, the sum and difference of the initial frequencies. The last two frequency couplings are often referred to as up- and down-conversion respectively.

It is clear from \cref{eq:MixPNL} that for the up- and down-conversion, the third wave frequency and -vector will be:
\begin{align}
  \label{qe:freqCond}
  \omega_3 = \omega_1 \pm \omega_2 \\
  \mathbf{k}_3 = \mathbf{k}_1 \pm \mathbf{k}_2.
\end{align}

Once the third wave is created, it will also couple back to
the two other waves, but these constraints secures that a neutral interaction is sustained through time and space.
This effect will add to the spectrum, however.

%In a simple model the up-conversion can be considered as a multiphoton excitation that emits new light at the sum frequency. The down-conversion as a excitation of the first photon energy and a deexcitation (or destructive interference) in the material corresponding to the second photon energy, then emitting the difference frequency.



\section{Numerical formulation of up-conversion}
\label{sec:mixing-numeric}
 
To get a numerical interpretation of the result, one can assume the
\cref{qe:freqCond} and inject the up-conversion part of \cref{eq:MixPNL} into the time dependent wave equation of each wave \cite[Equation 3.39]{shen}:

\begin{align*}
\paren{ \pdiff{}{z}  + \frac{1}{v_i^{g}} } \mathcal{E}_i \paren{z, t}
&= \frac{i 2 \pi \omega_i^{2}}{k c^2} \mathbf{P}^{NL} \paren{z,t} e^{i \paren{k_i z - \omega_{i}
    t}} \Rightarrow \\%\tag{3.39}\\
\paren{ \pdiff{}{z}  + \frac{1}{v_i^{g}} } \mathcal{E}_i \paren{z, t}
&= \frac{i \omega_i \chi^{(2)}}{2 n_i c} \mathcal{E}_j^{*}\mathcal{E}_k e^{i (k_3 - k_2 -k_1) z}, \text{where } i \neq j \neq k \\
% \paren{ \pdiff{}{z}  + \frac{1}{v_1^{g}} } \mathcal{E}_1 \paren{z, t}
% &= \frac{i \omega_1 \chi^{(2)}}{2 n_1 c} \mathcal{E}_2^{*}\mathcal{E}_3 e^{i (k_3 - k_2 -k_1) z
%     t} \\
% \paren{ \pdiff{}{z}  + \frac{1}{v_2^{g}} } \mathcal{E}_2 \paren{z, t}
% &= \frac{i \omega_2 \chi^{(2)}}{2 n_2 c} \mathcal{E}_1^{*}\mathcal{E}_3 e^{i (k_3 - k_2 -k_1) z
%     t} \\
%   \paren{ \pdiff{}{z}  + \frac{1}{v_3^{g}} } \mathcal{E}_3 \paren{z, t}
%   &= \frac{i \omega_3 \chi^{(2)}}{2 n_3 c} \mathcal{E}_1\mathcal{E}_2 e^{i (k_3 - k_2 -k_1) z
%     t}, \\
\end{align*}
where the identity $\omega_i = k_ic$ has been used, and also the rotating wave
approximation and slowly varying amplitude approximation. These are equivalent to
the coupled partial differential
equations  (2) in \cite{bakker} and could be numerically solved with the Runge-Kutta procedure
as described therein.

Typically this is done in KDP or KTP crystals such as the sum frequency conversion
done in \cite{sumFreq} where a 455 nm is made from a 807 nm and a 1062 source. Though only with an efficienty of $0.01\%$, which is assumed to be due to bad angle tuning. Efficiency's upto $93.7\%$ has been achieved by DTU Fotonik \cite{DTU}, which they use for optical coherence tomography, because it provides the freedom to use a tunable laser as one of the mixing frequencies, such that a chirpable output can be achieved.

Multiple wave mixing can be optained at higher orders of nonlinearity terms, such
that the third order gives rise to four-wave mixing and so on.


%%% Local Variables: 
%%% mode: latex
%%% TeX-master: "nonlinear"
%%% End: 

