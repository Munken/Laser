\chapter{Three-wave mixing}
\label{cha:mixing}


The Three-wave mixing is a property of the second order non-linearity of the polarity dependence of the electric field, $\mathcal{E}(t)$. 
It is a phenonemon where the two waves mix together to a third. 



\section{Mathematical formulation}
\label{sec:mixing-math}

To qualitatively examine the interaction between the two electromagnetic waves, of frequencies $\omega_1$ and $\omega_2$ with electric field amplitude $\mathcal{E}_1  \equiv \mathcal{E}(\omega_1)$ and $\mathcal{E}_1  \equiv \mathcal{E}(\omega_1)$ respectivily, a superposition of these monocromatic plane waves is composed: 
\[
\mathbf{\mathcal{E}}(t) = \Re (\mathcal{E}_1e^{i(\mathbf{k_1} \cdot \mathbf{r} - \omega_1 t)}+\mathcal{E}_2e^{i(\mathbf{k_2} \cdot \mathbf{r} \omega_2 t)}),
\]
and put into the second order part of the polarity expansion
\begin{align*}
     \mathbf{P}^{NL} & = \varepsilon_0 \chi^{(2)} \mathbf{E}^2(t) \\
&= \varepsilon_0 \chi^{(2)} 2 \Re \left[
|E_1|^2e^{i2(\mathbf{k_1} z - \omega_1t)}+|E_2|^2e^{i2(\mathbf{k_2} z - \omega_2t)} \right.\\
& +2E_1E_2e^{i((\mathbf{k_1} + \mathbf{k_2}) \cdot \mathbf{r} - (\omega_1+\omega_2)t)}\\
& +2E_1E_2^*e^{i((\mathbf{k_1} - \mathbf{k_2}) \cdot \mathbf{r} - (\omega_1-\omega_2)t}\\
& \left.+\left(|E_1|^2+|E_2|^2\right)e^{0}\right],
\end{align*}
which has frequencies $0$, which is uninteresting, $2\omega_1$ and $2\omega_2$, which are the original frequencies second harmonics, and more interresting the difference and sum of the initial frequencies. The last two frequeny couplings are typically refered to as down- and up-conversion respectively. 


\section{Numeric formulation}
\label{sec:mixing-numeric}

\begin{align}
P = 
\end{align}


Example KDP crystal

Notes:
four wave frequency wave mixing of higher order

%%% Local Variables: 
%%% mode: latex
%%% TeX-master: "nonlinear"
%%% End: 
