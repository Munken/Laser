\documentclass[a4paper,oneside,article]{memoir}
\usepackage[utf8]{inputenc}
\usepackage[T1]{fontenc}
\usepackage[english]{babel}
\usepackage{amsmath}
\usepackage{amssymb}
\usepackage{mathtools}
\usepackage{xparse}
\usepackage[draft]{fixme}

\usepackage{cleveref}
\usepackage{url}
\urlstyle{rm}

% Subfigure
\newsubfloat{figure}% Allow subfloats in figure environment


% Pæne fonte. Mums mums mums!
\usepackage[sf]{libertine}
\usepackage[sc]{mathpazo}
\linespread{1.06}
\usepackage{microtype}

% Fancy break med tre små stjerner, der signalerer et skifte, der ikke
% er start nok til at skulle have egen opgavebetegnelse.
\usepackage{fourier-orns}
\newcommand{\starbreak}{%
\fancybreak{\starredbullet\quad\starredbullet\quad\starredbullet}}

% BibLaTeX er nemmere at bruge end BibTeX og understøtter UTF-8 og
% sortering på andet end engelsk ved brug af backenden biber, der er
% standard.
\usepackage{csquotes}
\usepackage[
  backend=biber,
  %sortlocale=da_DK,
]{biblatex}
\addbibresource{laser.bib}
% Med article som option til memoir, så bliver references en /section
% og det vil vi ikke have. Derfor:
\defbibheading{bibliography}[\bibname]{%
  \chapter{#1}%
  \markboth{#1}{#1}}

% Afledte, partielle og ej, der alle tager to argumenter (som
% \frac). \diff er almindeligt afledt én gang og \ddiff er to gange -
% begge med oprejste(!) hårde d'er. \pdiff er partielt afledt én gang
% og \ppdiff to gange.
\newcommand\pdiff[3][\partial]{\frac{#1 #2}{#1 #3}}
\newcommand\ppdiff[3][\partial]{\frac{#1^2 #2}{#1 #3^2}}
\newcommand\ddiff[2]{\pdiff[\mathrm{d}]{#1}{#2}}
\newcommand\dddiff[2]{\ppdiff[\mathrm{d}]{#1}{#2}}

% Kommando til at lave parenteser på begge sider
\newcommand\paren[1]{\left(#1\right)}

% Hårdt og oprejst d til brug for integraler. \intd laver et halvt
% mellemrum før d'et og bruges normalt. \intd* laver ikke mellemrum og
% bruges efter fx hegn, der laver egne mellemrum.
\NewDocumentCommand{\intd}{s}{\IfBooleanTF{#1}{}{\,}\mathrm{d}}

% Random convenience.
\newcommand{\inv}{^{-1}}             % "I minus første"
\newcommand{\idx}[1]{_{\textup{#1}}} % Tekst-indeks med få tastetryk.
\newcommand{\order}[1]{^{\textup{(#1)}}} % Orden
\newcommand{\half}{\frac{1}{2}}

\hyphenation{Fi-gu-re
  fi-gu-re}


\begin{document}
\author{
  Anders Aspegren Søndergaard \\
  Kristoffer Theis Skalmstang \\
  Michael Munch \\
  Steffen Videbæk Fredsgaard \\
}
\title{Nonlinear optics}\date{\today}
\maketitle

\tableofcontents

\chapter{Introduction}
\label{cha:intro}

During the better part of the course we have studied linear phenomena,
that is the interaction between matter and light described by the wave
equation
\begin{align}
  \label{eq:wave-general}
  \nabla^2 \mathbf E - \frac{1}{c^2} \ppdiff{\mathbf E}{t}
  = \frac{1}{\epsilon_0c^2} \ppdiff{\mathbf P}{t},
\end{align}
where the polarization $\mathbf P$ is linear wrt. the electric field
$\mathbf E$, as described by $\mathbf P = \epsilon_0 \chi
\mathbf E$, where $\chi$ is electric susceptibility of the medium. The
course briefly touched upon nonlinear phenomena, where the polarization
is be taylor expanded as $\mathbf P = \epsilon_0 ( \chi\order1 \mathbf E +
\chi\order2 \mathbf E^2 + \dots )$.

In this project we describe and discuss two such effects: beam
self-focusing using the Kerr effect; and wave mixing.


\chapter{The Kerr effect}
\label{cha:kerr}

The Kerr effect is found in media, where the index of refraction is
dependent on the electric field of the propagating light. For real
life laser beams, where the (time avaraged) amplitude of electric
field depends on the position in the beam, this leads to a index of
refraction varying over the spacial position within the beam. Under
certain and experimentally obtainable conditions, this leads to
self-focusing of the beam within a passive medium.


\section{Mathematical formulation}
\label{sec:kerr-math}

As for second harmonic generation, which has been described in the
course, our model is a two-state atom and a monochromatic field with
semi-classical light/matter interaction. As in \cite{milonni}, the
frequency of monochromatic laser field is determined by the energies
of the two states of the atom, $\omega = \frac{E_2 - E_1}{\hbar}$,
leading to the expression
\begin{align}
  \label{eq:kerr-laser-field}
  E = \mathcal{E}(r, \omega) e^{-i \omega t},
\end{align}
for the electric field in a isotropic material with radial symmetry;
here $\mathcal{E}$ is the complex field amplitude. The polarization
can then be described similarily as $\mathcal{P}(r, \omega) e^{-i
  \omega t}$, where strength of the polarization $\mathcal{P}$ has
both a linear and a non-linear contribution.

In this formulation, the wave equation becomes
\begin{align}
  \label{eq:kerr-wave-eqn}
  \nabla^2 \mathcal{E} + \frac{\omega^2}{c^2} \mathcal{E}
  = \frac{1}{\epsilon_0 c^2} \mathcal{P}
\end{align}





%%% Local Variables: 
%%% mode: latex
%%% TeX-master: "nonlinear"
%%% End: 

\chapter{Three-wave mixing}
\label{cha:mixing}


The Three-wave mixing is a property of the second order non-linearity of the polarity dependence of the electric field, $\mathcal{E}(t)$. 
It is a phenomenon where the two waves mix together to a third. 



\section{Mathematical formulation}
\label{sec:mixing-math}

To qualitatively examine the interaction between the two electromagnetic waves, of frequencies $\omega_1$ and $\omega_2$ with electric field amplitude $\mathcal{E}_1  \equiv \mathcal{E}(\omega_1)$ and $\mathcal{E}_1  \equiv \mathcal{E}(\omega_1)$ respectively, a superposition of these monochromatic plane waves is composed: 
\[
\mathbf{\mathcal{E}}(t) = \Re (\mathcal{E}_1e^{i(\mathbf{k_1} \cdot \mathbf{r} - \omega_1 t)}+\mathcal{E}_2e^{i(\mathbf{k_2} \cdot \mathbf{r} \omega_2 t)}),
\]
and put into the second order part of the polarity expansion \cite{saleh}
\begin{align*}
     \mathbf{P}^{NL} & = \varepsilon_0 \chi^{(2)} \mathbf{E}^2(t) \\
&= \varepsilon_0 \chi^{(2)} 2 \Re \left[
\left(|E_1|^2+|E_2|^2\right)e^{0} + |E_1|^2e^{i2(\mathbf{k_1} z - \omega_1t)}+|E_2|^2e^{i2(\mathbf{k_2} z - \omega_2t)} \right.\\
& \left. +2E_1E_2e^{i((\mathbf{k_1} + \mathbf{k_2}) \cdot \mathbf{r} - (\omega_1+\omega_2)t)}  +2E_1E_2^*e^{i((\mathbf{k_1} - \mathbf{k_2}) \cdot \mathbf{r} - (\omega_1-\omega_2)t)} \right].
\end{align*}
Since this gives feedback to the electromagnetic field new frequencies are generated; $0$, the uninteresting one, $2\omega_1$ and $2\omega_2$, which are the original frequencies second harmonics, and more interesting, the sum and difference of the initial frequencies. The last two frequency couplings are often referred to as up- and down-conversion respectively.

It is clear that for the up- and down-conversion the third wave frequency will be:
\[
\omega_3 = \omega_1 \pm \omega_2
\]
and that a further phase matching constraint on the wave vector is necessary,
\[
\mathbf{k}_3 = \mathbf{k}_1 \pm \mathbf{k}_2,
\]
for a strong feedback in electric field change. Which phy





\section{Numeric formulation}
\label{sec:mixing-numeric}

\begin{align}
P = 
\end{align}


Example KDP crystal

Notes:
four wave frequency wave mixing of higher order

%%% Local Variables: 
%%% mode: latex
%%% TeX-master: "nonlinear"
%%% End: 







\nocite{*}
\printbibliography


\end{document}

%%% Local Variables: 
%%% mode: latex
%%% TeX-master: t
%%% End: 
