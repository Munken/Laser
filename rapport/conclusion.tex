\chapter{Conclusion}
\label{cha:conclusion}


The Kerr effect on the refractive index was worked out from the third nonlinear term of the polarization.
From this the self-focusing effect was could be explained by the in-homogeneous refractive index composition.
And the critical intensity for this effect to conquer diffractive spreading was presented for the cylindrical beam and the Gaussian beam, which was basis wave for the numerical work. 
Here the point of departure was the dimensionless waveequation, where the second order derivatives was approximated to by the central finite difference. Then using the Crank-Nicolson method the problem was reformulated the problem to a five diagonal matrix which should be solved for each time step. The implementation of the Crank-Nicolson was numerically unstable. Though a tendency of centered amplification was observed, eigther energy conservation broke down, or strong oscillations manifested at the center.


Since there was issues with the numerical solutions to the partial differential equations a qualitative explanation of the three wave mixing was presented; here the second order term couples the two waves together, which then couple back to the electric field creating a third wave, as the sum or difference of the initial frequencies, assuming that the phase matching constraint are fulfilled. Finally examples of the up-conversion in praxis with efficiency's up to $93.7\%$ are presented.


