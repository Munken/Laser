\chapter{Conclusion}
\label{cha:conclusion}


The kerr effect on the refractive index was worked out from the third nonlinear term of the polarization.
From this the self-fokusing effect was explainedcould be explaine by the inhomogeneous refractive index composition.
And the critical intensity for this effect to conquer diffractive index was presented for the cylendrical beam and the Gaussian beam, which was basis wave for the numerical work. 
Here 


Since there was issues with the numerical solutions to the partial differential equations a qualitative explaination of the three wave mixing was presented; here the second order term couples the two waves together, which then couple back to the electric field creating a third wave, as the sum or difference of the initial frequencies, assuming that the phase matching constraint are fulfilled. Finally examples of the up-conversion in praxis with efficiency's up to $93.7\%$ are presented.


